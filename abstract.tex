\ifdefined\build\else\input{head}\fi 

\begin{abstract}
This thesis addresses the problem of variable autonomy in teleoperated mobile robots. Variable autonomy refers to the approach of incorporating several different levels of autonomous capabilities (Level(s) of Autonomy (LOA)) ranging from pure teleoperation (human has complete control of the robot) to full autonomy (robot has control of every capability), within a single robot. Most robots used for demanding and safety critical tasks (e.g. search and rescue, hazardous environments inspection), are currently teleoperated in simple ways, but could soon start to benefit from variable autonomy. The use of variable autonomy would allow Artificial Intelligence (AI) control algorithms to autonomously take control of certain functions when the human operator is suffering a high workload, high cognitive load, anxiety, or other distractions and stresses. In contrast, some circumstances may still necessitate direct human control of the robot. More specifically, this thesis is focused on investigating the issues of dynamically changing LOA (i.e. during task execution) using either Human-Initiative (HI) or Mixed-Initiative (MI) control. MI refers to the peer-to-peer relationship between the robot and the operator in terms of the authority to initiate actions and LOA switches. HI refers to the human operators switching LOA based on their judgment, with the robot having no capacity to initiate LOA switches. A HI and a novel expert-guided MI controller are presented in this thesis. These controllers were evaluated using a multidisciplinary systematic experimental framework, that combines quantifiable and repeatable performance degradation factors for both the robot and the operator. The thesis presents statistically validated evidence that variable autonomy, in the form of HI and MI, provides advantages compared to only using teleoperation or only using autonomy, in various scenarios. Lastly, analyses of the interactions between the operators and the variable autonomy systems are reported. These analyses highlight the importance of personality traits and preferences, trust in the system, and the understanding of the system by the human operator, in the context of HRI with the proposed controllers.
\end{abstract}

\ifdefined\build\else\input{tail}\fi
